
% Default to the notebook output style

    


% Inherit from the specified cell style.




    
%\documentclass[11pt]{article}
\documentclass[a4paper, 12pt]{article}
\usepackage{fontspec} % loaded by polyglossia, but included here for transparency 
\usepackage{polyglossia}
\setmainlanguage{russian} 
\setotherlanguage{english}

\setmainfont[Ligatures=TeX]{Times New Roman}
\newfontfamily\cyrillicfont{Times New Roman}[Script=Cyrillic]
    

    
    \usepackage[T2A]{fontenc}			% кодировка
    % Nicer default font (+ math font) than Computer Modern for most use cases
%    \usepackage{mathpazo}

    % Basic figure setup, for now with no caption control since it's done
    % automatically by Pandoc (which extracts ![](path) syntax from Markdown).
    \usepackage{graphicx}
    % We will generate all images so they have a width \maxwidth. This means
    % that they will get their normal width if they fit onto the page, but
    % are scaled down if they would overflow the margins.
    \makeatletter
    \def\maxwidth{\ifdim\Gin@nat@width>\linewidth\linewidth
    \else\Gin@nat@width\fi}
    \makeatother
    \let\Oldincludegraphics\includegraphics
    % Set max figure width to be 80% of text width, for now hardcoded.
    \renewcommand{\includegraphics}[1]{\Oldincludegraphics[width=.8\maxwidth]{#1}}
    % Ensure that by default, figures have no caption (until we provide a
    % proper Figure object with a Caption API and a way to capture that
    % in the conversion process - todo).
    \usepackage{caption}
    \DeclareCaptionLabelFormat{nolabel}{}
    \captionsetup{labelformat=nolabel}

    \usepackage{adjustbox} % Used to constrain images to a maximum size 
    \usepackage{xcolor} % Allow colors to be defined
    \usepackage{enumerate} % Needed for markdown enumerations to work
    \usepackage{geometry} % Used to adjust the document margins
    \usepackage{amsmath} % Equations
    \usepackage{amssymb} % Equations
    \usepackage{textcomp} % defines textquotesingle
    % Hack from http://tex.stackexchange.com/a/47451/13684:
    \AtBeginDocument{%
        \def\PYZsq{\textquotesingle}% Upright quotes in Pygmentized code
    }
    \usepackage{upquote} % Upright quotes for verbatim code
    \usepackage{eurosym} % defines \euro
%    \usepackage[mathletters]{ucs} % Extended unicode (utf-8) support
    \usepackage{mathtext} 				% русские буквы в формулах
%    \usepackage[utf8]{inputenc}			% кодировка исходного текста
%    \usepackage[english,russian]{babel}	% локализация и переносы
    \usepackage{misccorr}				% настройки для соответствия правилам отечественной полиграфии
    
    \usepackage{fancyvrb} % verbatim replacement that allows latex
    \usepackage{grffile} % extends the file name processing of package graphics 
                         % to support a larger range 
    % The hyperref package gives us a pdf with properly built
    % internal navigation ('pdf bookmarks' for the table of contents,
    % internal cross-reference links, web links for URLs, etc.)
    \usepackage{hyperref}
    \usepackage{longtable} % longtable support required by pandoc >1.10
    \usepackage{booktabs}  % table support for pandoc > 1.12.2
    \usepackage[inline]{enumitem} % IRkernel/repr support (it uses the enumerate* environment)
    \usepackage[normalem]{ulem} % ulem is needed to support strikethroughs (\sout)
                                % normalem makes italics be italics, not underlines
    \usepackage{mathrsfs}
    

    
    
    % Colors for the hyperref package
    \definecolor{urlcolor}{rgb}{0,.145,.698}
    \definecolor{linkcolor}{rgb}{.71,0.21,0.01}
    \definecolor{citecolor}{rgb}{.12,.54,.11}

    % ANSI colors
    \definecolor{ansi-black}{HTML}{3E424D}
    \definecolor{ansi-black-intense}{HTML}{282C36}
    \definecolor{ansi-red}{HTML}{E75C58}
    \definecolor{ansi-red-intense}{HTML}{B22B31}
    \definecolor{ansi-green}{HTML}{00A250}
    \definecolor{ansi-green-intense}{HTML}{007427}
    \definecolor{ansi-yellow}{HTML}{DDB62B}
    \definecolor{ansi-yellow-intense}{HTML}{B27D12}
    \definecolor{ansi-blue}{HTML}{208FFB}
    \definecolor{ansi-blue-intense}{HTML}{0065CA}
    \definecolor{ansi-magenta}{HTML}{D160C4}
    \definecolor{ansi-magenta-intense}{HTML}{A03196}
    \definecolor{ansi-cyan}{HTML}{60C6C8}
    \definecolor{ansi-cyan-intense}{HTML}{258F8F}
    \definecolor{ansi-white}{HTML}{C5C1B4}
    \definecolor{ansi-white-intense}{HTML}{A1A6B2}
    \definecolor{ansi-default-inverse-fg}{HTML}{FFFFFF}
    \definecolor{ansi-default-inverse-bg}{HTML}{000000}

    % commands and environments needed by pandoc snippets
    % extracted from the output of `pandoc -s`
    \providecommand{\tightlist}{%
      \setlength{\itemsep}{0pt}\setlength{\parskip}{0pt}}
    \DefineVerbatimEnvironment{Highlighting}{Verbatim}{commandchars=\\\{\}}
    % Add ',fontsize=\small' for more characters per line
    \newenvironment{Shaded}{}{}
    \newcommand{\KeywordTok}[1]{\textcolor[rgb]{0.00,0.44,0.13}{\textbf{{#1}}}}
    \newcommand{\DataTypeTok}[1]{\textcolor[rgb]{0.56,0.13,0.00}{{#1}}}
    \newcommand{\DecValTok}[1]{\textcolor[rgb]{0.25,0.63,0.44}{{#1}}}
    \newcommand{\BaseNTok}[1]{\textcolor[rgb]{0.25,0.63,0.44}{{#1}}}
    \newcommand{\FloatTok}[1]{\textcolor[rgb]{0.25,0.63,0.44}{{#1}}}
    \newcommand{\CharTok}[1]{\textcolor[rgb]{0.25,0.44,0.63}{{#1}}}
    \newcommand{\StringTok}[1]{\textcolor[rgb]{0.25,0.44,0.63}{{#1}}}
    \newcommand{\CommentTok}[1]{\textcolor[rgb]{0.38,0.63,0.69}{\textit{{#1}}}}
    \newcommand{\OtherTok}[1]{\textcolor[rgb]{0.00,0.44,0.13}{{#1}}}
    \newcommand{\AlertTok}[1]{\textcolor[rgb]{1.00,0.00,0.00}{\textbf{{#1}}}}
    \newcommand{\FunctionTok}[1]{\textcolor[rgb]{0.02,0.16,0.49}{{#1}}}
    \newcommand{\RegionMarkerTok}[1]{{#1}}
    \newcommand{\ErrorTok}[1]{\textcolor[rgb]{1.00,0.00,0.00}{\textbf{{#1}}}}
    \newcommand{\NormalTok}[1]{{#1}}
    
    % Additional commands for more recent versions of Pandoc
    \newcommand{\ConstantTok}[1]{\textcolor[rgb]{0.53,0.00,0.00}{{#1}}}
    \newcommand{\SpecialCharTok}[1]{\textcolor[rgb]{0.25,0.44,0.63}{{#1}}}
    \newcommand{\VerbatimStringTok}[1]{\textcolor[rgb]{0.25,0.44,0.63}{{#1}}}
    \newcommand{\SpecialStringTok}[1]{\textcolor[rgb]{0.73,0.40,0.53}{{#1}}}
    \newcommand{\ImportTok}[1]{{#1}}
    \newcommand{\DocumentationTok}[1]{\textcolor[rgb]{0.73,0.13,0.13}{\textit{{#1}}}}
    \newcommand{\AnnotationTok}[1]{\textcolor[rgb]{0.38,0.63,0.69}{\textbf{\textit{{#1}}}}}
    \newcommand{\CommentVarTok}[1]{\textcolor[rgb]{0.38,0.63,0.69}{\textbf{\textit{{#1}}}}}
    \newcommand{\VariableTok}[1]{\textcolor[rgb]{0.10,0.09,0.49}{{#1}}}
    \newcommand{\ControlFlowTok}[1]{\textcolor[rgb]{0.00,0.44,0.13}{\textbf{{#1}}}}
    \newcommand{\OperatorTok}[1]{\textcolor[rgb]{0.40,0.40,0.40}{{#1}}}
    \newcommand{\BuiltInTok}[1]{{#1}}
    \newcommand{\ExtensionTok}[1]{{#1}}
    \newcommand{\PreprocessorTok}[1]{\textcolor[rgb]{0.74,0.48,0.00}{{#1}}}
    \newcommand{\AttributeTok}[1]{\textcolor[rgb]{0.49,0.56,0.16}{{#1}}}
    \newcommand{\InformationTok}[1]{\textcolor[rgb]{0.38,0.63,0.69}{\textbf{\textit{{#1}}}}}
    \newcommand{\WarningTok}[1]{\textcolor[rgb]{0.38,0.63,0.69}{\textbf{\textit{{#1}}}}}
    
    
    % Define a nice break command that doesn't care if a line doesn't already
    % exist.
    \def\br{\hspace*{\fill} \\* }
    % Math Jax compatibility definitions
    \def\gt{>}
    \def\lt{<}
    \let\Oldtex\TeX
    \let\Oldlatex\LaTeX
    \renewcommand{\TeX}{\textrm{\Oldtex}}
    \renewcommand{\LaTeX}{\textrm{\Oldlatex}}
    % Document parameters
    % Document title
    \title{Описание расчётного модуля}
    
    
    
    
    

    % Pygments definitions
    
\makeatletter
\def\PY@reset{\let\PY@it=\relax \let\PY@bf=\relax%
    \let\PY@ul=\relax \let\PY@tc=\relax%
    \let\PY@bc=\relax \let\PY@ff=\relax}
\def\PY@tok#1{\csname PY@tok@#1\endcsname}
\def\PY@toks#1+{\ifx\relax#1\empty\else%
    \PY@tok{#1}\expandafter\PY@toks\fi}
\def\PY@do#1{\PY@bc{\PY@tc{\PY@ul{%
    \PY@it{\PY@bf{\PY@ff{#1}}}}}}}
\def\PY#1#2{\PY@reset\PY@toks#1+\relax+\PY@do{#2}}

\expandafter\def\csname PY@tok@w\endcsname{\def\PY@tc##1{\textcolor[rgb]{0.73,0.73,0.73}{##1}}}
\expandafter\def\csname PY@tok@c\endcsname{\let\PY@it=\textit\def\PY@tc##1{\textcolor[rgb]{0.25,0.50,0.50}{##1}}}
\expandafter\def\csname PY@tok@cp\endcsname{\def\PY@tc##1{\textcolor[rgb]{0.74,0.48,0.00}{##1}}}
\expandafter\def\csname PY@tok@k\endcsname{\let\PY@bf=\textbf\def\PY@tc##1{\textcolor[rgb]{0.00,0.50,0.00}{##1}}}
\expandafter\def\csname PY@tok@kp\endcsname{\def\PY@tc##1{\textcolor[rgb]{0.00,0.50,0.00}{##1}}}
\expandafter\def\csname PY@tok@kt\endcsname{\def\PY@tc##1{\textcolor[rgb]{0.69,0.00,0.25}{##1}}}
\expandafter\def\csname PY@tok@o\endcsname{\def\PY@tc##1{\textcolor[rgb]{0.40,0.40,0.40}{##1}}}
\expandafter\def\csname PY@tok@ow\endcsname{\let\PY@bf=\textbf\def\PY@tc##1{\textcolor[rgb]{0.67,0.13,1.00}{##1}}}
\expandafter\def\csname PY@tok@nb\endcsname{\def\PY@tc##1{\textcolor[rgb]{0.00,0.50,0.00}{##1}}}
\expandafter\def\csname PY@tok@nf\endcsname{\def\PY@tc##1{\textcolor[rgb]{0.00,0.00,1.00}{##1}}}
\expandafter\def\csname PY@tok@nc\endcsname{\let\PY@bf=\textbf\def\PY@tc##1{\textcolor[rgb]{0.00,0.00,1.00}{##1}}}
\expandafter\def\csname PY@tok@nn\endcsname{\let\PY@bf=\textbf\def\PY@tc##1{\textcolor[rgb]{0.00,0.00,1.00}{##1}}}
\expandafter\def\csname PY@tok@ne\endcsname{\let\PY@bf=\textbf\def\PY@tc##1{\textcolor[rgb]{0.82,0.25,0.23}{##1}}}
\expandafter\def\csname PY@tok@nv\endcsname{\def\PY@tc##1{\textcolor[rgb]{0.10,0.09,0.49}{##1}}}
\expandafter\def\csname PY@tok@no\endcsname{\def\PY@tc##1{\textcolor[rgb]{0.53,0.00,0.00}{##1}}}
\expandafter\def\csname PY@tok@nl\endcsname{\def\PY@tc##1{\textcolor[rgb]{0.63,0.63,0.00}{##1}}}
\expandafter\def\csname PY@tok@ni\endcsname{\let\PY@bf=\textbf\def\PY@tc##1{\textcolor[rgb]{0.60,0.60,0.60}{##1}}}
\expandafter\def\csname PY@tok@na\endcsname{\def\PY@tc##1{\textcolor[rgb]{0.49,0.56,0.16}{##1}}}
\expandafter\def\csname PY@tok@nt\endcsname{\let\PY@bf=\textbf\def\PY@tc##1{\textcolor[rgb]{0.00,0.50,0.00}{##1}}}
\expandafter\def\csname PY@tok@nd\endcsname{\def\PY@tc##1{\textcolor[rgb]{0.67,0.13,1.00}{##1}}}
\expandafter\def\csname PY@tok@s\endcsname{\def\PY@tc##1{\textcolor[rgb]{0.73,0.13,0.13}{##1}}}
\expandafter\def\csname PY@tok@sd\endcsname{\let\PY@it=\textit\def\PY@tc##1{\textcolor[rgb]{0.73,0.13,0.13}{##1}}}
\expandafter\def\csname PY@tok@si\endcsname{\let\PY@bf=\textbf\def\PY@tc##1{\textcolor[rgb]{0.73,0.40,0.53}{##1}}}
\expandafter\def\csname PY@tok@se\endcsname{\let\PY@bf=\textbf\def\PY@tc##1{\textcolor[rgb]{0.73,0.40,0.13}{##1}}}
\expandafter\def\csname PY@tok@sr\endcsname{\def\PY@tc##1{\textcolor[rgb]{0.73,0.40,0.53}{##1}}}
\expandafter\def\csname PY@tok@ss\endcsname{\def\PY@tc##1{\textcolor[rgb]{0.10,0.09,0.49}{##1}}}
\expandafter\def\csname PY@tok@sx\endcsname{\def\PY@tc##1{\textcolor[rgb]{0.00,0.50,0.00}{##1}}}
\expandafter\def\csname PY@tok@m\endcsname{\def\PY@tc##1{\textcolor[rgb]{0.40,0.40,0.40}{##1}}}
\expandafter\def\csname PY@tok@gh\endcsname{\let\PY@bf=\textbf\def\PY@tc##1{\textcolor[rgb]{0.00,0.00,0.50}{##1}}}
\expandafter\def\csname PY@tok@gu\endcsname{\let\PY@bf=\textbf\def\PY@tc##1{\textcolor[rgb]{0.50,0.00,0.50}{##1}}}
\expandafter\def\csname PY@tok@gd\endcsname{\def\PY@tc##1{\textcolor[rgb]{0.63,0.00,0.00}{##1}}}
\expandafter\def\csname PY@tok@gi\endcsname{\def\PY@tc##1{\textcolor[rgb]{0.00,0.63,0.00}{##1}}}
\expandafter\def\csname PY@tok@gr\endcsname{\def\PY@tc##1{\textcolor[rgb]{1.00,0.00,0.00}{##1}}}
\expandafter\def\csname PY@tok@ge\endcsname{\let\PY@it=\textit}
\expandafter\def\csname PY@tok@gs\endcsname{\let\PY@bf=\textbf}
\expandafter\def\csname PY@tok@gp\endcsname{\let\PY@bf=\textbf\def\PY@tc##1{\textcolor[rgb]{0.00,0.00,0.50}{##1}}}
\expandafter\def\csname PY@tok@go\endcsname{\def\PY@tc##1{\textcolor[rgb]{0.53,0.53,0.53}{##1}}}
\expandafter\def\csname PY@tok@gt\endcsname{\def\PY@tc##1{\textcolor[rgb]{0.00,0.27,0.87}{##1}}}
\expandafter\def\csname PY@tok@err\endcsname{\def\PY@bc##1{\setlength{\fboxsep}{0pt}\fcolorbox[rgb]{1.00,0.00,0.00}{1,1,1}{\strut ##1}}}
\expandafter\def\csname PY@tok@kc\endcsname{\let\PY@bf=\textbf\def\PY@tc##1{\textcolor[rgb]{0.00,0.50,0.00}{##1}}}
\expandafter\def\csname PY@tok@kd\endcsname{\let\PY@bf=\textbf\def\PY@tc##1{\textcolor[rgb]{0.00,0.50,0.00}{##1}}}
\expandafter\def\csname PY@tok@kn\endcsname{\let\PY@bf=\textbf\def\PY@tc##1{\textcolor[rgb]{0.00,0.50,0.00}{##1}}}
\expandafter\def\csname PY@tok@kr\endcsname{\let\PY@bf=\textbf\def\PY@tc##1{\textcolor[rgb]{0.00,0.50,0.00}{##1}}}
\expandafter\def\csname PY@tok@bp\endcsname{\def\PY@tc##1{\textcolor[rgb]{0.00,0.50,0.00}{##1}}}
\expandafter\def\csname PY@tok@fm\endcsname{\def\PY@tc##1{\textcolor[rgb]{0.00,0.00,1.00}{##1}}}
\expandafter\def\csname PY@tok@vc\endcsname{\def\PY@tc##1{\textcolor[rgb]{0.10,0.09,0.49}{##1}}}
\expandafter\def\csname PY@tok@vg\endcsname{\def\PY@tc##1{\textcolor[rgb]{0.10,0.09,0.49}{##1}}}
\expandafter\def\csname PY@tok@vi\endcsname{\def\PY@tc##1{\textcolor[rgb]{0.10,0.09,0.49}{##1}}}
\expandafter\def\csname PY@tok@vm\endcsname{\def\PY@tc##1{\textcolor[rgb]{0.10,0.09,0.49}{##1}}}
\expandafter\def\csname PY@tok@sa\endcsname{\def\PY@tc##1{\textcolor[rgb]{0.73,0.13,0.13}{##1}}}
\expandafter\def\csname PY@tok@sb\endcsname{\def\PY@tc##1{\textcolor[rgb]{0.73,0.13,0.13}{##1}}}
\expandafter\def\csname PY@tok@sc\endcsname{\def\PY@tc##1{\textcolor[rgb]{0.73,0.13,0.13}{##1}}}
\expandafter\def\csname PY@tok@dl\endcsname{\def\PY@tc##1{\textcolor[rgb]{0.73,0.13,0.13}{##1}}}
\expandafter\def\csname PY@tok@s2\endcsname{\def\PY@tc##1{\textcolor[rgb]{0.73,0.13,0.13}{##1}}}
\expandafter\def\csname PY@tok@sh\endcsname{\def\PY@tc##1{\textcolor[rgb]{0.73,0.13,0.13}{##1}}}
\expandafter\def\csname PY@tok@s1\endcsname{\def\PY@tc##1{\textcolor[rgb]{0.73,0.13,0.13}{##1}}}
\expandafter\def\csname PY@tok@mb\endcsname{\def\PY@tc##1{\textcolor[rgb]{0.40,0.40,0.40}{##1}}}
\expandafter\def\csname PY@tok@mf\endcsname{\def\PY@tc##1{\textcolor[rgb]{0.40,0.40,0.40}{##1}}}
\expandafter\def\csname PY@tok@mh\endcsname{\def\PY@tc##1{\textcolor[rgb]{0.40,0.40,0.40}{##1}}}
\expandafter\def\csname PY@tok@mi\endcsname{\def\PY@tc##1{\textcolor[rgb]{0.40,0.40,0.40}{##1}}}
\expandafter\def\csname PY@tok@il\endcsname{\def\PY@tc##1{\textcolor[rgb]{0.40,0.40,0.40}{##1}}}
\expandafter\def\csname PY@tok@mo\endcsname{\def\PY@tc##1{\textcolor[rgb]{0.40,0.40,0.40}{##1}}}
\expandafter\def\csname PY@tok@ch\endcsname{\let\PY@it=\textit\def\PY@tc##1{\textcolor[rgb]{0.25,0.50,0.50}{##1}}}
\expandafter\def\csname PY@tok@cm\endcsname{\let\PY@it=\textit\def\PY@tc##1{\textcolor[rgb]{0.25,0.50,0.50}{##1}}}
\expandafter\def\csname PY@tok@cpf\endcsname{\let\PY@it=\textit\def\PY@tc##1{\textcolor[rgb]{0.25,0.50,0.50}{##1}}}
\expandafter\def\csname PY@tok@c1\endcsname{\let\PY@it=\textit\def\PY@tc##1{\textcolor[rgb]{0.25,0.50,0.50}{##1}}}
\expandafter\def\csname PY@tok@cs\endcsname{\let\PY@it=\textit\def\PY@tc##1{\textcolor[rgb]{0.25,0.50,0.50}{##1}}}

\def\PYZbs{\char`\\}
\def\PYZus{\char`\_}
\def\PYZob{\char`\{}
\def\PYZcb{\char`\}}
\def\PYZca{\char`\^}
\def\PYZam{\char`\&}
\def\PYZlt{\char`\<}
\def\PYZgt{\char`\>}
\def\PYZsh{\char`\#}
\def\PYZpc{\char`\%}
\def\PYZdl{\char`\$}
\def\PYZhy{\char`\-}
\def\PYZsq{\char`\'}
\def\PYZdq{\char`\"}
\def\PYZti{\char`\~}
% for compatibility with earlier versions
\def\PYZat{@}
\def\PYZlb{[}
\def\PYZrb{]}
\makeatother


    % Exact colors from NB
    \definecolor{incolor}{rgb}{0.0, 0.0, 0.5}
    \definecolor{outcolor}{rgb}{0.545, 0.0, 0.0}



    
    % Prevent overflowing lines due to hard-to-break entities
    \sloppy 
    % Setup hyperref package
    \hypersetup{
      breaklinks=true,  % so long urls are correctly broken across lines
      colorlinks=true,
      urlcolor=urlcolor,
      linkcolor=linkcolor,
      citecolor=citecolor,
      }
    % Slightly bigger margins than the latex defaults
    
    \geometry{verbose,tmargin=1in,bmargin=1in,lmargin=1in,rmargin=1in}
    
    \usepackage{indentfirst}
    \parindent=1.25cm 

    \begin{document}
    
    
    \maketitle
    
    

    

\section{Введение}

    Правильное проектирование насосной станции позволяет сэкономить большое
количество электроэнергии при эксплуатации. Для этого, как правило, насосы подбирают 
по максимальной точке. Программное обеспечение Grundfos позволяет дополнительно 
оценить энергозатраты по нескольким рабочим точкам.

    В документе содержится информация об инструменте для оптимизации
конструкции и работы насосной станции для произвольного количества рабочих точек,
в том числе и для случаев, когда необходимо использовать насосы разных размеров.


\section{Методика}

    Насосная станция представлена в виде набора насосных агрегатов с
произвольными характеристиками. Никаких ограничений на "одинаковость"
насосов не накладывается, насосные агрегаты включены параллельно.


\subsection{Основы}

    Каждый насосный агрегат представляет собой набор из непосредственно
насоса, двигателя и частотного преобразователя. Каждый механизм имеет
свои характеристики, алгоритм расчёта которых описан ниже.


\subsubsection{Насос}

    Каждый насос описывается характеристической кривой (зависимость напора
от расхода) и кривой КПД (зависимость от расхода).

    Кривая характеристики насоса $H(Q)$ задается в табличном формате парами
значений \((Q, H)\), которые потом аппроксимируются полиномом второго порядка \cite{Shankar}:

\begin{equation}\label{eq:HQ_curve}
	H(Q) = C_0 \cdot \omega^2 - C_1 \cdot Q \cdot \omega - C_2 \cdot Q^2,
\end{equation}

\noindent где $H$ - напор, $Q$ - расход, $\omega$ - нормированная частота вращения
(при номинальных параметрах $\omega = 1$), а $C_0, C_1, C_2$ - коэффициенты, которые
получаются при аппроксимации исходных табличных данных.

Отдельно задаются специфичные параметры \(Q_{\min}\) и \(Q_{\max}\),
которые определяют минимально и максимально допустимые расходы при
номинальной частоте вращения. Расход, напор и мощность $P$ на разных
частотах рассчитываются через законы подобия:

\begin{equation}\label{eq:affinity}
	\frac{Q_2}{Q_1} = \frac{n_2}{n_1} = \omega, \frac{H_2}{H_1} = \left(\frac{n_2}{n_1}\right)^2 = \omega^2, \frac{P_2}{P_1} = \left(\frac{n_2}{n_1}\right)^3 = \omega^3.
\end{equation}

В диапазоне работы КПД насоса $\eta_P$ аппроксимируется параболой. Если известно только
значение максимального КПД \((Q_{BEP}, \eta_{BEP})\), кривая задается
формулой \cite{Chantasiriwan}:

\begin{equation}
	\eta_P(Q) = a \cdot Q^2 + b \cdot Q,
\end{equation}

\noindent коэффициенты которой находятся из следующей системы уравнений:

\begin{equation}
\label{eq:eff_bep}
	\left\{
    \begin{aligned}
        &a \cdot Q^2_{BEP} + b \cdot Q_{BEP} - \eta_{BEP} = 0 \\
        &2a \cdot Q_{BEP} + b = 0
    \end{aligned}
	\right.
\end{equation}

\noindent Второе уравнение этой системы получено из условия равенства нулю производной в точке
максимума \((Q_{BEP}, \eta_{BEP})\). Если известно две точки \((Q_{BEP}, \eta_{BEP})\) и \((Q_2, \eta_2)\),
\(Q_2 > Q_{BEP}\), то кривая задается следующей системой

\begin{equation}
	\eta_P(Q) = 
    \begin{cases}
        a \cdot Q^2 + b \cdot Q, &\text{если }  Q < Q_{BEP} \\
        a_2 \cdot Q^2 + b_2 \cdot Q + c_2, &\text{если }  Q \ge Q_{BEP}
    \end{cases}
\end{equation}

\noindent где коэффициенты \(a\) и \(b\) находятся из системы уравнений (\ref{eq:eff_bep}), а
остальные из следующей:

\begin{equation}
    \left\{
    \begin{aligned}
            &a_2 \cdot Q^2_{BEP} + b_2 \cdot Q_{BEP} + c_2 - \eta_{BEP} = 0 \\
            &2a_2 \cdot Q_{BEP} + b_2 = 0 \\
            &a_2 \cdot Q^2_2 + b_2 \cdot Q_2 + c_2 - \eta_2 = 0
	\end{aligned}
    \right.
\end{equation}

\noindent Зависимость гидравлического КПД насоса от скорости вращения рассчитывается, используя \cite{Sárbu}

\begin{equation}
	\eta_P = \eta_2 = 1 - (1 - \eta_1) \cdot \left(\frac{n_1}{n_2}\right)^{0,1} = 1 - (1 - \eta_1) \cdot \left(\frac{1}{\omega}\right)^{0,1}.
\end{equation}

    Гидравлическая мощность:

\begin{equation}
	P_H = \rho gQH,
\end{equation}

\noindent где $\rho$ - плотность жидкости, $g$ - ускорение свободного падения.

Мощность на валу насоса будет равна:

\begin{equation}
	P_S = \frac{P_H}{\eta_P}.
\end{equation}

\noindent Эту мощность мы считаем полезной мощностью на валу двигателя.

\subsubsection{Двигатель}

Для двигателя указываются номинальная мощность \(P_{NOM}\) и значения КПД при нагрузках \(100\,\%\) и \(75\,\%\) - \(\eta_{M,100}\) и \(\eta_{M,75}\), соответственно. Нагрузка двигателя:

\begin{equation}
	L = \frac{P_S}{P_{NOM} / \eta_{M,100}}.
\end{equation}

\noindent Согласно стандарту IEC 60034-31 \cite{IEC} можно рассчитать КПД двигателя, при
любом значении нагрузки по формуле:

\begin{equation}
  \begin{aligned}
        &\nu_L = \frac{ \left(\frac{1}{\eta_{M,100}}-1\right) - 0,75 \left(\frac{1}{\eta_{M,75}}-1\right)}{0,4375} \\
        &\nu_0 = \left(\frac{1}{\eta_{M,100}}-1\right)-\nu_L \\
        &\eta_M = \frac{1}{1+ (\nu_0 / L) + \nu_L \cdot L}.
    \end{aligned}
\end{equation}

\noindent Отсюда, потребляемая мощность двигателя:

\begin{equation}
	P_M = \frac{P_S}{\eta_M}.
\end{equation}


\subsubsection{Частотный преобразователь}

    Считаем, что КПД частотного преобразователя на всем диапазоне нагрузок
постоянен. Принимаем, что \(\eta_{VFD} = const\), а потребляемая мощность электрической сети:

\begin{equation}
	P_E = \frac{P_M}{\eta_{VFD}}.
\end{equation}


\subsubsection{Насосный агрегат}

    Общий КПД насосного агрегата складывается из КПД составляющих частей -
насоса, двигателя и частотного преобразователя:

\begin{equation}\label{eq:eff_total}
	\eta_{TOT} = \frac{P_H}{P_E} = \eta_P \cdot \eta_M \cdot \eta_{VFD}.
\end{equation}

\subsection{Алгоритм}

Оптимизация выполняется для каждой рабочей точки, которую может создать насосная станция, используя заданное пользователем разрешение \(Q_{step} \times H_{step}\). Размер шага зависит от требуемой точности и влияет на время вычислений и объём памяти. Задача оптимизации для каждой рабочей точки \((Q_i, H_j)\) выглядит
следующим образом:

\begin{equation}
\underset{\vec{f}_{i,j} \in X_{i,j}}{\min} P_E\left(Q_i,H_j,\vec{f}_{i,j}\right),
\end{equation}

\noindent где \(\vec{f}_{i,j}\) - это вектор частот каждого насоса, а пространство
поиска \(X_{i,j}\) включает все возможные комбинации, которые дают
итоговую рабочую точку \((Q_i, H_j)\). Оптимизация выполняется прямым
поиском (перебором) всех возможных значений и выбора среди них
минимального.

Сначала находим оптимальные режимы работы каждого насоса во всех рабочих точках \((Q_i, H_j)\) в диапазоне частот [25, 50] Гц.

\begin{enumerate}

\item
Для фиксированной точки \((Q_i, H_j)\) проходим по всем возможным частотам с шагом 0.02 Гц. 
Для рассматриваемой рабочей точки выбирается та частота, которая дает наибольший КПД (\ref{eq:eff_total}).

\item
Минимальный и максимальный допустимые напоры \(H_{\min}\) и \(H_{\max}\), а также минимальный и максимальный допустимые расходы \(Q_{\min}\) и \(Q_{\max}\) рассчитываются на основе характеристики насоса (\ref{eq:HQ_curve}) и законов подобия (\ref{eq:affinity}).

\item
Результаты расчёта хранятся в двух специальных матрицах.
\(\mathbf{F}\) содержит оптимальную частоту, а \(\mathbf{H}\) содержит
КПД насосного агрегата для соответствующей рабочей точки и частоты. Элементы матриц,
которые представляют недопустимые рабочие точки, устанавливаются в 0.

\begin{equation}
	\mathbf{F} = \left[\begin{matrix}
		f_{Q_1,H_1} & f_{Q_2,H_1} & \dots & f_{Q_m,H_1} \\
		f_{Q_1,H_2} & f_{Q_2,H_2} & \dots & f_{Q_m,H_2} \\ 
		\vdots & \vdots & \ddots & \vdots \\
		f_{Q_1,H_n} & f_{Q_2,H_n} & \dots & f_{Q_m,H_n} \\
	\end{matrix}\right], 
	\mathbf{H} = \left[\begin{matrix}
		\eta_{Q_1,H_1} & \eta_{Q_2,H_1} & \dots & \eta_{Q_m,H_1} \\
		\eta_{Q_1,H_2} & \eta_{Q_2,H_2} & \dots & \eta_{Q_m,H_2} \\ 
		\vdots & \vdots & \ddots & \vdots \\
		\eta_{Q_1,H_n} & \eta_{Q_2,H_n} & \dots & \eta_{Q_m,H_n} \\
	\end{matrix}\right].
\end{equation}
\end{enumerate}

    Далее рассматриваются все возможные комбинации насосов. Для каждой
комбинации насосов необходимо проделать следующее: 

\begin{enumerate}

\item
Определяем разрешённый диапазон напора. Минимальный и максимальный напор диапазона
рассчитывается следующим образом (\(n\) - количество насосов, работающих
в комбинации):

\begin{equation}
	\begin{aligned}
		H_{\min} &= \max\{H_{\min,1}, H_{\min,2}, \cdots, H_{\min,n}\}, \\
		H_{\max} &= \min\{H_{\max,1}, H_{\max,2}, \cdots, H_{\max,n}\}.
	\end{aligned}
\end{equation}


\item
Проходим по разрешённому диапазону напора \([H_{\min}, H_{\max}]\) c
  шагом \(H_{step}\). Напор в каждой точке обозначим как \(H_j\). Для
  напора \(H_j\) находим максимальный возможный расход для выбранной комбинации:

\begin{equation}
	Q_{\max}(H_j) = \sum_{k=1}^{n} Q_{\max}^k(H_j),
\end{equation}

\noindent где $Q_{\max}^k(H_j)$ - максимальный расход для $k$-го насоса в комбинации для напора $H_j$.

\item
  С шагом \(Q_{step}\) проходим по всему диапазону
  расхода \([0, Q_{\max}(H_j)]\), который даёт исследуемая комбинация
  насосов. В каждой точке расхода \(Q_i\) вычисляем все возможные
  комбинации расходов для насосов, работающих в комбинации. Для одного и
  того же значения расхода \(Q_i\) множество значений таких расходов в
  зависимости от количества насосов \(n\):

\begin{equation}
	Q_i = \sum_{k=1}^{n} q_k.
\end{equation}


\item
  КПД каждого насоса $\eta_k$ определяется из массива рабочих режимов насоса
  \(\mathbf{H}\) для расхода $q_k$. Общий КПД при этом рассчитывается для общего расхода
  \(Q_i\).

\begin{equation}
	\eta_{Q_i, H_j} = \frac{\sum_{k=1}^{n} q_k}{\sum_{k=1}^{n} \frac{q_k}{\eta_k}} = \prod_{k=1}^{n} \eta_k \cdot \frac{\sum_{k=1}^{n} q_k}{\sum_{k=1}^{n} (q_k \cdot \prod_{p \ne k}^{n} \eta_p)}.
\end{equation}

Здесь $q_k$ - значение расхода для k-го насоса из комбинации при напоре $H_j$, а $\eta_k$ - значение КПД для k-го насоса при расходе $q_k$ и напоре $H_j$. Расход $q_k$ находится в диапазоне $[0, Q_i]$, а конкретные значения выбираются по макисмуму общего КПД:
 
\begin{equation}
	\max {\eta_{Q_i, H_j}}, \text{при условии: } q_k \in [0, Q_i], Q_i = \sum_{k=1}^{n} q_k, H_j = const.
\end{equation}
 
\item
  Из всех вариантов КПД для рабочей точки \((Q_i, H_j)\) наибольшее
  значение сохраняется в матрице результатов \(\mathbf{R}\). Количество
  насосов, которое даёт этот лучший результат КПД, сохранется в другой
  матрице \(\mathbf{C}\). Конечный результат - два массива, которые
  покрывают все возможные режимы работы насосной станции для выбранной
  комбинации работы насосов. Каждый элемент представляет область,
  определенную шагом \(Q_{step} \times H_{step}\):

\begin{equation}
	\mathbf{C} = \left[\begin{matrix}
		c_{Q_1,H_1} & c_{Q_2,H_1} & \dots & c_{Q_m,H_1} \\
		c_{Q_1,H_2} & c_{Q_2,H_2} & \dots & c_{Q_m,H_2} \\ 
		\vdots & \vdots & \ddots & \vdots \\
		c_{Q_1,H_n} & c_{Q_2,H_n} & \dots & c_{Q_m,H_n} \\
	\end{matrix}\right], 
	\mathbf{R} = \left[\begin{matrix}
		\eta_{Q_1,H_1} & \eta_{Q_2,H_1} & \dots & \eta_{Q_m,H_1} \\
		\eta_{Q_1,H_2} & \eta_{Q_2,H_2} & \dots & \eta_{Q_m,H_2} \\ 
		\vdots & \vdots & \ddots & \vdots \\
		\eta_{Q_1,H_n} & \eta_{Q_2,H_n} & \dots & \eta_{Q_m,H_n} \\
	\end{matrix}\right].
\end{equation}

\end{enumerate}


\subsection{Применение}

Возможно несколько сценариев работы с методикой, один из которых - использование для энергоаудита.
В этом случае рассчитывается текущая эффективность насосной станции и оценивается потенциал энеросбережения
при использовании различных насосных агрегатов. На выходе мы получаем расчёт окупаемости при внесении тех или иных изменений в алгоритмы и (или) оборудование.

\subsubsection{Расчёт энергопотребления}

    Исходные данные напора и расхода зависят от времени и для определения
требуемых пар значений \((Q_i, H_j)\) измеренная зависимость расхода
квантуется по уровню с шагом \(Q_{step}\), а зависимость напора с шагом
\(H_{step}\). После чего определяются соответствующие индексы в матрицах
результатов \(\mathbf{C}\) и \(\mathbf{R}\), откуда можно легко
рассчитать оптимальную электрическую потребляемую активную мощность для
каждой комбинации насосных агрегатов:

\begin{equation}
	P_E(Q_i, H_j) = \frac{P_H(Q_i, H_j)}{\eta_{Q_i, H_j}} = \frac{\rho gQ_iH_j}{\eta_{Q_i, H_j}}.
\end{equation}

\noindent Значения матрицы оптимального количества насосов в каждой рабочей точке
используется для определения границ переключения между насосами и
гистерезиса переключения.


\subsubsection{NPSH}

Низкое давление на всасе насоса и большая скорость вращения насоса приводят к понижению давления на лопатках насоса. Если давление жидкости при этом ниже значения давления насыщенных паров, жидкость закипает. Мельчайшие пузырьки пара жидкости двигаются вдоль лопаток рабочего колеса и в области высокого давления пузырьки быстро схлопываются. Образовавшиеся при схлопывании звуковые волны воздействуют на внутреннюю поверхность насоса, что может привести к серьезной эрозии рабочего колеса. Описанное явление называется кавитация. Таким образом, если мы хотим эффективно перекачивать жидкость, нужно сохранять её в жидком состоянии. Одним из параметров, который влияет на правильность эксплуатации насосных агрегатов, является высота столба жидкости над всасывающим патрубком насоса (NPSH).

Требуемое значение NPSH ($\text{NPSH}{_\text{R}}$, required) зависит от конструкции насоса. Когда жидкость проходит через всасывающий патрубок насоса и попадает на направляющий аппарат рабочего колеса, скорость жидкости увеличивается, а давление падает. Также возникают потери давления из-за турбулентности и неровности потока жидкости. Центробежная сила лопаток рабочего колеса также увеличивает скорость и уменьшает давление жидкости. $\text{NPSH}{_\text{R}}$ - необходимый подпор на всасывающем патрубке насоса, чтобы компенсировать все потери давления в насосе и удержать жидкость выше уровня давления насыщенных паров, и ограничить потери напора, возникающие в результате кавитации на уровне 3\%. Такой запас на падение напора - общепринятый критерий $\text{NPSH}{_\text{R}}$, принятый для облегчения расчета. Большинство насосов с низкой всасывающей способностью могут работать с низким или минимальным запасом по $\text{NPSH}{_\text{R}}$, что серьезно не сказывается на сроке их эксплуатации. $\text{NPSH}{_\text{R}}$ зависит от скорости и производительности насосов. Производители насосов предоставляют информацию о характеристике $\text{NPSH}{_\text{R}}$, кторую необходимо задавать в табличном формате парами значений \((Q, \text{NPSH}{_\text{R}})\). Эти значения потом аппроксимируются полиномом второго порядка:

\begin{equation}\label{eq:NPSH_curve}
	\text{NPSH}{_\text{R}}(Q) = D_0 \cdot \omega^2 - D_1 \cdot Q \cdot \omega - D_2 \cdot Q^2,
\end{equation}

\noindent где $D_0, D_1, D_2$ - коэффициенты, которые получаются при аппроксимации исходных табличных данных. При изменении частоты вращения насоса для вычисления $\text{NPSH}{_\text{R}}$ применяются законы подобия (\ref{eq:affinity}), как и для напора:

\begin{equation}\label{eq:affinity1}
	\frac{\text{NPSH}_2}{\text{NPSH}_1} = \left(\frac{n_2}{n_1}\right)^2 = \omega^2.
\end{equation}

Допустимый NPSH ($\text{NPSH}{_\text{A}}$, allowable) - является характеристикой системы, в которой работает насос. Это разница между атмосферным давлением, высотой всасывания насоса и давлением насыщенных паров:

\begin{equation}
	\text{NPSH}{_\text{A}} = \frac{P_{\text{атм}}}{\rho g} + h_S - \frac{V_S^2}{2g} - \Delta h_f - \Delta h_m - \frac{P_{\text{нас}}}{\rho g} ,
\end{equation}

\noindent где $P_{\text{атм}}$ - атмосферное давление, $h_S$ - максимальная высота всасывания, $V_S$ - скорость жидкости на всасе (3-е слагаемое - это динамический напор во всасывающем трубопроводе), $\Delta h_f$ - потери на трение во всасывающем трубопроводе при требуемой производительности насоса, $\Delta h_m$ - незначительные потери на входе в насос, $P_{\text{нас}}$ - давление насыщенных паров жидкости при максимальной рабочей температуре.

В реальной системе $\text{NPSH}{_\text{A}}$ определяется с помощью показаний манометра, установленного на стороне всасывания насоса. Применяется следующая формула:

\begin{equation}
	\text{NPSH}{_\text{A}} = \frac{P_{\text{атм}}}{\rho g} + \frac{V_S^2}{2g} - \frac{P_{\text{нас}}}{\rho g} \pm G_r,
\end{equation}

\noindent где $G_r$ - показания манометра на всасывании насоса, выраженные в метрах, взятые с плюсом (+) , если давление выше атмосферного и с минусом (-), если ниже, с поправкой на осевую линию насоса.


Чтобы предотвратить кавитацию для стандартных насосов с низкой всасывающей способностью, необходимо обеспечить, чтобы $\text{NPSH}{_\text{A}}$ системы был выше, чем $\text{NPSH}{_\text{R}}$ насоса. Насосы с высокой всасывающей способностью требуют запаса для $\text{NPSH}{_\text{R}}$. Стандарт Гидравлического Института (ANSI/HI 9.6.1) предлагает увеличивать $\text{NPSH}{_\text{A}}$ в 1.2 - 2.5 раза для насосов с высокой и очень высокой всасывающей способностью. Поэтому, в общем случае, при работе в допустимом диапазоне рабочих характеристик должно выполняться следующее условие:

\begin{equation}
	\text{NPSH}{_\text{A}} \ge \text{NPSH}{_\text{R}} + \text{запас}.
\end{equation}


\noindent 



    % Add a bibliography block to the postdoc
    
\begin{thebibliography}{00} % Список литературы

	\bibitem{Shankar}
	Shankar V. K. A. et al. A comprehensive review on energy efficiency enhancement initiatives in centrifugal pumping system //Applied Energy. – 2016. – Т. 181. – С. 495-513.
	
	\bibitem{Chantasiriwan}
	Chantasiriwan S. Performance of variable-speed centrifugal pump in pump system with static head //International Journal of Power and Energy Systems. – 2013. – Т. 33. – №. 1.
	
	\bibitem{Sárbu}
	Sárbu I., Borza I. Energetic optimization of water pumping in distribution systems //Periodica Polytechnica Mechanical Engineering. – 1998. – Т. 42. – №. 2. – С. 141-152.
	
	\bibitem{IEC}
	IEC I. E. C. 60034-31: Rotating electrical machines-Part 31. – 2009.
	
\end{thebibliography}
    
    
    
    
    \end{document}
